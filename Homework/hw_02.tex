% Homework 1
\documentclass[12pt, letterpaper]{article}

\usepackage[top=1in, bottom=1.5in, left=.5in, right=.5in]{geometry}
\usepackage[colorlinks=true,urlcolor=blue]{hyperref}

% Force pdflatex to use correct paper size.
\special{papersize=8.5in,11in}
\setlength{\pdfpageheight}{\paperheight}
\setlength{\pdfpagewidth}{\paperwidth}

\begin{document}

\begin{center}
  {\LARGE \textbf{PHYS-5391 Assignment 2}}\\
  %{\large Due date: 2 October 2017, 5:00PM}\\
\end{center}

\vspace{0.5cm}

In class, we wrote a Python module that contains a function to open and parse
a simple file that contained observed solar wind conditions at the L-1 point.
On the Canvas site, the example file is available for you to download.
Get it and put it in a spot you'll want it, then
make sure that Python can find it by editing the ``python path''.  This is
the list of directories that Python searches for when you try to import
modules.  If you're using the BASH shell, you can edit the python path
by setting the following environment variable:
\begin{verbatim}
export PYTHONPATH=/your/path/here
\end{verbatim}
To make this change permanent, place that line into your shell configuration
file (i.e., {\tt ~/.bashrc} or {\tt ~/.profile}.  Grab the IMF file from Canvas
and make sure that this module can successfully read it.

Solve each of the following and submit well-commented source code files for
each.  I will test your source files on my own machine.  Include enough 
comments so that I know how and why you're performing each step, 
ensure each line is briefly justified (e.g., ``Create file object in read
mode''), and answer any questions in the sections below.
\textbf{No \LaTeX document is required for this problem set.}  Include any
discussion in the Python source file via comments, doc strings, or in the
output written to the screen.

\begin{enumerate}
\item Use the module to load the example data.  Calculate 
  $|\mathbf{B}|$ and $|\mathbf{V}|$ over the time period covered by the file.
  To ensure that your results are correct, save them to file in a manner that
  will make the calculation easy to check by hand.  Create one file
  for $|\mathbf{B}|$ and $|\mathbf{V}|$; use a format similar to this:
\begin{verbatim}
Bx     By     Bz    |B|
-06.47 +00.83 +14.4 +15.81
\end{verbatim}
With such a format, it is easy to plug $B_X$, $B_Y$, and $B_Z$ into a
calculator to see that  the resulting $|\mathbf{B}|$ has been calculated
correctly.
Be sure to have one line for each line in the original data file, 
{\tt imf\_aug2005 .dat}.  Pick a reasonable number of significant digits, etc. 

Writing files in Python is straightforward.  Simply create a file object
using 'write' mode instead of 'read' mode.  Then, use the file object's
{\tt write} method to print a single line to file. Finally, close the file
using the {\tt close} method.  It will look something like this:
\begin{verbatim}
f = open('test.txt', 'w')
f.write('Here is a test line.\n')
f.write('Here is a number: {:.7f}\n'.format(3.14))
f.close()
\end{verbatim}
\emph{Always} end each line with the newline character ({\tt \textbackslash n}).
Use either the {\tt format} string object method or
\href{https://docs.python.org/3/tutorial/inputoutput.html#tut-f-strings}{``f-strings'' in Python 3}.
Study up on \href{https://docs.python.org/3/library/string.html#formatspec}{the format sub-language (i.e., format codes)} to learn how to precisely format each line.

\item Calculate the mean of $|\mathbf{B}|$ and $|\mathbf{V}|$ over the time
  period covered by the file.  To do this, use either Numpy's {\tt mean}
  function or the Numpy array object method with the same name.
  Use the {\tt print} function and an intelligent format code to write the
  answer to screen in an easy-to-read manner.  

\item $D_{ST}$ is an extraordinarily useful index in space weather.  It
  indicates the strength of the geomagnetic ring current by measuring the
  reduction of the Earth's magnetic field resulting from the diamagnetic
  current.  A more negative value indicates a stronger ring 
  current.  Standardized, definitive $D_{ST}$ can be found at
  \href{http://wdc.kugi.kyoto-u.ac.jp/}{Kyoto's World Data Center}.
  Their ASCII file format, however, is less than ideal.

  On Canvas, locate {\tt Dst\_July2000.dat}, an example $D_{ST}$ data file.
  You will find it in Files/data\_files.  The format of the file is described 
  \href{http://wdc.kugi.kyoto-u.ac.jp/dstae/format/dstformat.html}{here}.
  One issue is that many of the values run together with out a space or tab
  delimiter, meaning that careful parsing of each line is required.  While
  brute-force will work, a more elegant solution can be created with careful
  string indexing within a {\tt for} loop.

  Open this file using Python and load the $D_{ST}$ into a 1-dimension Numpy
  array This is going to take a bit of thought, so be sure to
  \emph{write out every step in plain English BEFORE you start coding.}
  Using the {\tt min}, {\tt max}, {\tt mean}, {\tt median}, and {\tt std}
  Numpy functions, explore the data.  Print the details to screen in an
  easy-to-read manner.  You may be asked to read and manipulate similar files
  in the future, so it is to your advantage to make this reader robust and
  easy-to-use.

  \textbf{Bonus:} Create an array of date-time objects to match
  your $D_{ST}$  values.  For such a task, you just might find Python's
  \href{https://docs.python.org/3/tutorial/datastructures.html#list-comprehensions}{list comprehension syntax}
  particularly elegant.

\end{enumerate}

\end{document}
