% Homework 1
\documentclass[12pt, letterpaper]{article}

\usepackage[top=1in, bottom=1.5in, left=.5in, right=.5in]{geometry}
\usepackage[colorlinks=true,urlcolor=blue]{hyperref}
\usepackage{enumerate}

% Force pdflatex to use correct paper size.
\special{papersize=8.5in,11in}
\setlength{\pdfpageheight}{\paperheight}
\setlength{\pdfpagewidth}{\paperwidth}

\begin{document}

\begin{center}
  {\LARGE \textbf{PHYS-5391 Assignment 4}}\\
\end{center}

%\vspace{0.5cm}

The theory of evolution by natural selection has a history rife with resistance.
Despite rich, overwhelming, evidence-based support, acceptance
is not necessarily widespread, 
\href{http://en.wikipedia.org/wiki/Level_of_support_for_evolution#United_States}
{especially amongst the American public.}  A specific example comes from
genetics: the genetic code unambiguously supports the idea of evolution from
common ancestry.  Evolution deniers point out that the chemical
ordering of a given DNA strand, giving rise to genetic information, 
\href{http://en.wikipedia.org/wiki/Hoyle%27s_fallacy}{could not have
arisen from pure chance}, as the odds of the molecules falling into the correct
order randomly is ludicrously small.  On its face, this statement is true, but
it neglects that evolution is random variation \emph{guided} by natural
selection.

To illustrate this point, Richard Dawkins created what is popularly known
as \href{http://en.wikipedia.org/wiki/Weasel_program}{the weasel program.}
The concept starts with the colloquialism, ``a million monkeys with a million
type writers will produce Shakespeare given enough time.''  Simplifying
to a single line from a single Shakespearian play, {\tt METHINKS IT IS LIKE
A WEASEL}, it is trivial to create a computer program that randomly spits out
28 characters (limited to capital letters and spaces) and test to see if it
matches the target phrase.  Waiting for such a program to create a match takes
a very long time.  However, if we introduce some basic concepts from 
evolutionary biology, the code converges to the target phrase very quickly.

The weasel program starts with a 28-length random set of letters using the 
limit of only capital letters and spaces.  
For each iteration, or \emph{generation}, the initial phrase, or \emph{parent}
phrase, produces 100 \emph{offspring}.  These are copies where random mutations
are introduced: each letter has a 4\% chance of changing into a different
letter.  Then, each is compared against the target phrase.  The offspring 
with the most number of exact matches (i.e., correct letters in the correct
position) is considered the most \emph{fit}.  The process then repeats with
the fittest offspring becoming the new parent.  Using such an algorithm, the
target phrase can be constructed with a surprisingly low number of generations.
The lesson expressed is that evolution via natural selection is \textbf{not}
a process of pure chance, but guided by a population's environment.


Your goal should be a function that, given number of offspring and the
mutation rate as inputs, returns the number of generations required to arrive
at the target phrase.  \href{http://docs.python.org/2/library/random.html}
{Python's {\tt random} module will be very useful here.}  For example, you
\emph{could} use {\tt random.random()} as a ``dice roll'' whose result, if
larger than your mutation rate, would indicate that a certain letter should
be changed.  Further, you could create a list of all available characters
and use {\tt random.choice(seq)} to randomly return one of those characters.
These are just suggestions; there are many ways to approach this problem.
\textbf{It is paramount that you write down, in plain text, what you want
  your program to do before you start coding.}  If you neglect this step, you
will spend a lot of extra time writing code that reaches no clear goal.
In your \LaTeX write up, start with a description of your program.  Defend
your choice of programming strategy.  Using the mutation rate and generation
size listed above, how fast does your code typically converge on the target
phrase?  

Explore the two degrees of freedom for this program: the generation size and
mutation rate.  How does the convergence time (in terms of number of generations
) change with these parameters?  Does the program always converge?  Use words,
quantitative analysis, and even plots to support your conclusions.  If you only
submit a working code and a thorough description of your code, you will, 
\emph{at best}, receive a 3.  If you express imagination and thoroughness in 
your exploration of the code's results, you are almost guaranteed a 4 or 5.



\end{document}
