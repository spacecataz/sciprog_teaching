% Homework 1
\documentclass[12pt, letterpaper]{article}

\usepackage[top=1in, bottom=1.5in, left=.5in, right=.5in]{geometry}
\usepackage[colorlinks=true,urlcolor=blue,citecolor=blue]{hyperref}
\usepackage{enumerate}
\usepackage{natbib}
\usepackage{amsmath}

% Force pdflatex to use correct paper size.
\special{papersize=8.5in,11in}
\setlength{\pdfpageheight}{\paperheight}
\setlength{\pdfpagewidth}{\paperwidth}

\begin{document}

\begin{center}
  {\LARGE \textbf{PHYS-5391 Assignment 3}}\\
\end{center}

As discussed in Assignment 2, $D_{ST}$ is a valuable geomagnetic index of 
magnetospheric activity.  It is a measure of the deviation from quiet levels
of the Earth's magnetic field, measured in nano-Tesla ($nT$, sometimes called
``gammas'' in older literature).  A more negative $D_{ST}$ indicates stronger
activity.  $D_{ST}$ has a clear connection to solar wind
and interplanetary magnetic field (IMF) values.
The exact relationship between the two
has been a topic of study for decades.  In this assignment, you are going to
investigate both $D_{ST}$ and solar wind observations to see if you can make
any connections for yourself.

Download {\tt Dst\_July2000.dat} and {\tt imf\_jul2000.dat} from the Canvas
web site (in the ``data files'' folder found in the ``Files'' link).  These
files contain data from the famous ``Bastille Day'' solar storm, which arrived
at Earth on July 15th.  Use
Python to explore these files.  Create plots that show both $D_{ST}$ and one
or more of the variables inside of the solar wind file (remember that these
include the three components of the interplanetary magnetic field, three
components of the velocity, plasma temperature, and number density).
On your own,
explore the data until you find some qualitative connection between one (or
more) of the solar wind values and the observed $D_{ST}$ index.  Be sure to
focus on the storm itself, and not just the whole interval in the files (which
may or may not cover just the event.)

Once you think you have a story to tell about the connection between the
solar wind, IMF, and the $D_{ST}$ for this event, construct a \LaTeX document
that shares this story.  Include \textbf{polished, focused, and clear} plots
that support your words.  Make sure that anyone who sees your plots will
come to the same conclusion you do.  You will not be graded on if you are
correct in your assessment.  Rather, the key grading criteria will be the
clarity with which you present your argument.

You will turn in the following items:
\begin{itemize}
\item All Python source code files and data files required to reproduce
  the plots that you will include in your final write-up. Per the usual, your code
  should be clear, well-commented, and self-documented.  \emph{Every line that does
  anything plot-related should be commented!}  \textbf{Turn this portion in
    via a github repository.}
\item A final PDF that presents your conclusions both visually (i.e., one or
  more plots) and verbally.  Describe your plot in full and how it led you
  to your conclusions.  Include a section that defends your choice of plot
  style and why you made the plot the way you did.
  \textbf{Turn in this document via Canvas.}
\end{itemize}

To reiterate, the goal of this assignment is to get you to use plots to
explore data as well as illustrate conclusions of your investigations.  Create
plots that are a communication tool!  Better grades will be awarded for
clear plots with good, easy-to-read labels; good color schemes and styles;
annotations or other creative aspects.  Your write-up will be an important
part of this assignment, too.  A clear write-up with unambiguous plots is
the formula for success.

\end{document}
