% Homework 1
\documentclass[12pt, letterpaper]{article}

\usepackage[top=1in, bottom=1.5in, left=.5in, right=.5in]{geometry}

% Force pdflatex to use correct paper size.
\special{papersize=8.5in,11in}
\setlength{\pdfpageheight}{\paperheight}
\setlength{\pdfpagewidth}{\paperwidth}

\begin{document}

\begin{center}
  {\LARGE \textbf{PHYS 5391 Assignment 1}}\\
  {\large Due date: 11 September 2020, 5:00PM}\\
\end{center}

Most of the work involved with using \LaTeX \ is required up front:
it takes time
to create a template that will serve as a basis for future documents.  Hence,
like many things, the first time is the worst time (in terms of investment of
time and total effort). 
\emph{The goal of this assignment is to create a \LaTeX \ 
document that will serve as a template for your future homework write-ups.}

\begin{enumerate}
  \item If you do not have a \LaTeX \ interpreter installed, get one.
  \item \label{part2} From the Canvas page, download the entire contents of the 
    ``Modules/LaTeX Lecture'' folder.  Compile the source file (example.tex)
    into a PDF either via the Makefile (type {\tt make} at the command prompt)
    or with individual calls to \LaTeX and bibtex.  The manual commands are:
    \begin{verbatim}
      $ pdflatex example.tex
      $ pdflatex example.tex
      $ bibtex example
      $ pdflatex example.tex\end{verbatim}
      ..where the dollar sign indicates a Unix terminal command prompt.
      Alternatively, you may use TeXshop or Overleaf, but I recommend trying
      to compile the source from the command line at least once.
  \item \label{part3} Study both the PDF and example.tex source file.
  \item Create a \LaTeX \ document with the required elements listed below.
    In your source file, comment each major command set telling me why you
    are using that command and those specific arguments.  For example,
    what do the options for the {\tt \textbackslash documentclass} do, and why
    did you select them?  If you are importing packages, what does each
    package do?  When you use a {\tt \textbackslash begin} command, what 
    environment are you entering and why?  You will \emph{not} need to 
    comment your future \LaTeX \ documents.

    \textbf{Deliverables:} Submit your commented document source and a 
    compiled PDF document.
\end{enumerate}

Required components in your document:
\begin{itemize}
  \item A title page with the homework number, your name, and date.
  \item A table of contents.
  \item A section dedicated to your current programming experience.  Discuss
    what languages you have used in the past, what types of tasks you are
    comfortable performing, and what languages and tasks you would like to 
    learn.  If there are any texts that you found particularly helpful and
    would strongly recommend, be sure to cite them in your bibliography.
  \item Describe your programming environment in terms of operating system,
    text/code editor, and access to a command line terminal.  I want to know
    \emph{how} you'll be doing work for this class.  Are you using a personal
    Windows laptop or a Linux desktop?  Do you have access to the appropriate
    tools?
  \item A section where you select your favorite physical law/equation, list
    it explicitly, and describe in words what it means.  Describe each term in
    detail.  Include a table that lists each variable and its meaning.  Ensure
    that your equation has a number and that you reference it properly, e.g.
    ``Equation 1 shows that...''.  Use the {\tt \textbackslash ref} syntax such
    that if I add an equation randomly in your document, it will not 
    ruin your referencing.
  \item A section where you comment on the example document you studied for
    Questions \ref{part2} and \ref{part3}.  Was it clear?  Was it accurate?
    Was it complete?  Your input will be used to refine it.
  \item A section where you select three or more scholarly articles you have
    read and \emph{briefly} summarize them (one or two sentences per article).
    Be sure to properly cite each article in your bibliography!
  \item A section where you select a random picture that you obtain from
    the Internet and include it in the document.  Caption it, then include
    a paragraph that describes and references it properly (e.g., ``Figure 1
    illustrates how a dog would answer a phone...'').
  \item Perform some task \emph{not covered} in the example document.  For
    example, split your document into two columns, insert some colored text
    in the document, create clickable-hyperlinks, or customize the document's
    layout.  You will need to research what to do and how to do it.
    Include a section that says specifically what task you performed and how you
    were able to achieve it.  I want you to teach me how to do what you did.
  \item A bibliography page with at least three references
    included.  Use bibtex to simplify citations.  Use the Natbib package
    inside your source file.
\end{itemize}


\end{document}
