 % Homework 1
\documentclass[12pt, letterpaper]{article}

\usepackage[top=1in, bottom=1.5in, left=.5in, right=.5in]{geometry}
\usepackage[colorlinks=true,urlcolor=blue,citecolor=blue]{hyperref}
\usepackage{enumerate}
\usepackage{natbib}
\usepackage{amsmath}

% Force pdflatex to use correct paper size.
\special{papersize=8.5in,11in}
\setlength{\pdfpageheight}{\paperheight}
\setlength{\pdfpagewidth}{\paperwidth}

\begin{document}

\begin{center}
  {\LARGE \textbf{CLaSP 605-001 Assignment 5}}\\
  {\large Due date: Friday, 10 November 2017, 5:00PM}\\
\end{center}

As discussed in previous assignments, $D_{ST}$ is a valuable geomagnetic index
of magnetospheric activity.  It is a measure of the deviation from quiet levels
of the Earth's magnetic field, measured in nano-Tesla ($nT$, sometimes called
``gammas'' in older literature).  The ring current causes a decrease in the
terrestrial field, so a strongly negative $D_{ST}$ indicates a strong ring
current.  Other current systems, such as magnetopause currents, can also
manifest in $D_{ST}$, but the ring current appears to be the largest
contributor.  Because the ring current, and therefore $D_{ST}$, is an indicator
of storm strength, there have been many attempts to tie this value to 
upstream solar conditions and produce a predictive formula.

One such formula is the \textit{Burton Equation} \citep{Burton:1975}:
\begin{equation}
  \frac{d D_{ST}}{dt}=-aD_{ST}+ab\sqrt{P_{SW}(t)}-ac+F(E_{Y})+
  b\frac{d\sqrt{P_{SW}(t)}}{dt}
\end{equation}

\noindent ...where the function $F(E_{Y})$ is given by,

\begin{equation}
  \label{eq:F}
  F(E_{Y}) = 
  \begin{cases}
    0 & \text{if $E_Y<0.5\frac{mV}{m}$}\\
    d(E_Y-0.5) & \text{if $E_Y\geq 0.5\frac{mV}{m}$}
  \end{cases}
\end{equation}

\noindent The terms and factors are described below; see the full
text of \citet{Burton:1975} for derivation. With a set of upstream drivers,
it is possible to predict $D_{ST}$ by numerically integrating this ODE.
A tool for doing this, written in Python, can be found on Canvas in the
\emph{Files/Python Notes} folder.
It uses the \href{http://lpsa.swarthmore.edu/NumInt/NumIntFourth.html}
{Runge-Kutta method}, which is useful to know.

\begin{table}[!h]
  \begin{center}
    \begin{tabular}{c|c|l}
    Variable & Value & Explanation\\
    \hline\hline
    $E_Y$   & $v_x B_z$  & Solar wind electric field, in $mV/m$\\
    $F(E_Y)$ & Eq. \ref{eq:F} & Impact of plasma sheet injections\\
    $P_{SW}$ & $\rho v^2$ & Solar wind dynamic pressure, in $eV/cm^3$\\
    $a$ & $3.6\times 10^{-5}s^{-1}$ & Decay time constant of $D_{ST}$\\
    $b$ & $0.20nT(eV/cm^3)^{-1/2}$ & $D_{ST}$ sensitivity to $P_{SW}$\\
    $c$ & $20nT$ & Contribution of quiet time currents \\
    $d$ & $-1.5\times 10^{-3} nT*m/mV/s$ & Fitting constant for $F(E_Y)$\\
    \end{tabular}
    \caption{Description of terms within the Burton equation.}
    \label{tab:1}
  \end{center}
\end{table}

The parameters in Table \ref{tab:1} were selected to give best-fit results for
7 different geomagnetic storms.  This is a very limited set of data, so it
remains unknown (at least, to \emph{you}) how well this equation can actually
predict $D_{ST}$ on a day-to-day basis.
\textbf{Your goal for this assignment is to test the Burton equation's
  performance using a whole year's worth of IMF data.}

Your task is to integrate the Burton Equation over the entire year of 2003.
 Save the results of
the integration and sort the data such that there is one ASCII output file
per month.  Additionally, create a ``summary plot'' of Burton $D_{ST}$ versus
observed $D_{ST}$, again, one per month.  Come to some conclusions about
the long-term performance of the Burton Equation.  Are there particular storms
that stand out in terms of strong or poor reproduction?  Given the generic
knowledge that a space weather event is about to encounter the Earth, would
you trust the Burton Equation to give you an idea of the storm's strength?

Obtaining a year's worth of IMF data is not easy.
Fortunately, you have found the 
\href{http://www-personal.umich.edu/~dwelling/imf_2003/}
{perfect source of data online}, including IMF data in the familiar format
and observed $D_{ST}$ for the entire year.
Unfortunately, the IMF data is proliferated amongst 365 separate files, reducing
the feasibility of manually performing the above task.  Additionally, the 
$D_{ST}$ data is in a single file.  Upon seeing this data source, it occurs to
you that a script could be written to yield all of the results very, 
very quickly.

Perform the year-long analysis of the Burton Equation as described above.
You may use \emph{any method you choose} to complete the task, but you will
receive zero points if the method is not reproduceable on any computer within
a short amount of time.  This means that if you decide manually downloading
every file is the most efficient way, you will need to come to my office and
download each file at the arbitrary moment when I wish to grade your homework.
It is very likely that I will become bored of watching you do this and decide
that you should receive zero points.

\vspace{.5cm}
{\large Code Requirements:}
\noindent
\begin{enumerate}
\item Obtain all input data required to do the analysis.
\item For each month in the year 2003, create an ascii file of the predicted
  $D_{ST}$ as calculated by the Burton equation.
\item For each month in the year 2003, produce a summary plot that illustrates
  how well the Burton equation performs compared to observed $D_{ST}$.
\item Create some \emph{quantitative} measure of performance.  Examples include
  the various forms of
  \href{https://en.wikipedia.org/wiki/Root-mean-square_deviation}
       {root-mean-squared-error}.  Print this value to screen.
\end{enumerate}

\vspace{.5cm}
{\large What to turn in:}
\noindent
\begin{enumerate}
\item \textbf{Everything required to reproduce your results except for
the data.}  I must be able to recreate your results, so be sure to document
  your code in the write-up and in your comments and docstrings.  Make your code
  easy to run!
\item Your summary plots independent of your write-up.
\item Your write up, including a discussion of your methodology and the
results.  While including every summary plot in your write-up is unnecessary,
some ``highlights'' supporting your conclusions are strongly encouraged.
\item Include your quantitative analysis values in your write-up.  Discuss
  what they tell you about the Burton equation.
\end{enumerate}

\bibliographystyle{plainnat}
\bibliography{dtw_bib}

\end{document}
