\documentclass[11pt, letterpaper]{article}

\usepackage{anysize}
\usepackage[colorlinks=true,urlcolor=blue,citecolor=blue]{hyperref}

\marginsize{.5in}{.5in}{.5in}{.5in}

\begin{document}
% Force pdflatex to use correct paper size.
\special{papersize=8.5in,11in}
\setlength{\pdfpageheight}{\paperheight}
\setlength{\pdfpagewidth}{\paperwidth}

\begin{center}
  {\LARGE \textbf{Fortran 90: A Brief Introduction}\\}
  {\large D. T. Welling, \today}
\end{center}

\vspace{0.5cm}
\begin{center}
  \emph{“I don't know what the language of the year 2000 will look like, but I know it will be called Fortran.”}

    —Tony Hoare, winner of the 1980 Turing Award, in 1982.
\end{center}
\vspace{0.5cm}

Fortran is a powerful and \emph{fast} programming language for doing
scientific computations.
Though Fortran tends to elicit memories of the archaic Fortran77 standard, 
modern versions (Fortran90, 95, 2008 and 2018) are modern languages
that embrace modern paradigms including modularization, object-oriented
approaches, and concurrent programming.  Fortran remains popular in scientific
programming because of its speed, its inherent arrays and mathematical
capabilities.  

This tutorial will get you started with the basics
of the language syntax.  To do the tasks listed below, you will need a
text editor (I recommend \href{https://www.gnu.org/software/emacs/}{Emacs}
and a compiler.  Many factors can go into choice of compiler; for this
tutorial, the biggest factor is \emph{cost}.  The
\href{https://gcc.gnu.org/fortran/}{GNU Fortran compiler} (a.k.a. gfortran)
is free and widely available via package managers and software repositories.
We'll be using that for the remainder of this exercise.  Additionally,
while there are \href{https://www.google.com/search?q=fortran+ide&oq=fortran+ide&aqs=chrome..69i57j0l6j0i20i263.1596j0j4&sourceid=chrome&ie=UTF-8}
  {several Interactive Development Enviroments (IDEs) available
    for Fortran}, we'll be using the usual Linux command line interface.

\section{Hello World \& Compilation}

Let us begin at the common starting point: \emph{Hello World}.  Type the
following into a raw text file.  Name this file \texttt{hello\_world.f90}.
For Fortran 90 source code, use the file extension \texttt{.f90}.

\begin{verbatim}
program hello_world
  ! Basic hello world program.
  implicit none
  
  integer :: i=2
  
  write(*,*) "Hello World"
  write(*,*) "Here is an integer:", i

  i = i + 3.14
  write(*,*) "Is this correct? i + pi = ", i

  stop
end program hello_world
\end{verbatim}

Right away we see the basic elements of a Fortran program.  We start by
defining the name of the program and end with an {\tt end} statement.  This
``closes'' the program section of the file.  {\tt end} is a fundamental
statement in Fortran; it closes loops, {\tt if} statements, subroutines, etc.
We also see that comments are denoted with a leading exclamation point.
Be sure to comment your code well.  Next, there is \texttt{implicit none}.
\textbf{ALWAYS include \texttt{implicit none} at the beginning of your code.}
We'll talk more about that shortly.  Next, we declare a variable.  Fortran
is \emph{statically typed}, which means that once you declare a variable, it
cannot change its type.  An integer will remain an integer, even if you add
$\pi$ to it.  We then see some \texttt{write} statements that print information
to screen.  Finally, we see the \texttt{stop} statement, which tells the
computer to stop running the program.

A quick note concerning the \texttt{stop} statement: In situations where it is the
final statement, as in the above program, it is optional.  It is required in
other situations where you are stopping the program before reaching the end
of the file.  In real world situations, you rarely need the \texttt{stop}
statement.  Except in error-handling contexts, it is more often a sign of
sloppy programming rather than a situation of requirement.  In some contexts,
especially that of parallel programming, you want to avoid it.  Therefore,
if you are considering reaching for a \texttt{stop} statement, think twice, as it
is most often not necessary.

It is notable that our code is nicely spaced out.  All commands
between the start and end of our program are tabbed over slightly to make it
easier to see what is ``within'' our program.  This is not a requirement of
Fortran90; it is a \emph{free-form} language, in contrast to Python
(with its required tab-delimiting of code blocks) or earlier versions of
Fortran.  However, maintaining tabs for code blocks is strongly recommended.
It keeps your code easy to read and follow.  Doing so is widely accepted to
the point that many text editors will do this spacing for you.

We want to run this program, but we can't.  Fortran is a compiled language,
not an interpreted language.  This means that there is an extra step between
writing our code and executing the commands.  To compile with gfortran,
navigate to the same folder as your source code and type the following:
\begin{verbatim}
gfortran hello_world.f90
\end{verbatim}

If you look in the working directory, you'll now find an executable file
named \texttt{a.out}.  This is our program!  Run it by typing its name at the
command prompt and pressing enter.  You should see the following:

\begin{verbatim}
 Hello World
 Here is an integer:           2
 Is this correct? i + pi =           5
\end{verbatim}


At this point, a more friendly author would offer a tepid congratulations on
writing your first Fortran program.  Such patronizing is avoided here.

Some quick notes on this output: the formatting seems quite arbitrary, with our
integers hanging out in limbo away from our sentences.  Also, $2+\pi \neq 5$!
What gives with the math?  Remember that Fortran is \emph{statically typed}!  When
we did our addition, $2+\pi$ was converted into an integer before the result was
stored into a variable that is explicitly an integer.  Be careful with
\emph{mixed mode arithmatic} in statically-typed languages.

\section{The \texttt{read} \texttt{write} Statements}

Let's dive further into Fortran by exploring how to interact with the user.
This is absolutely critical because, unlike interpreted languages, there is
no easy way to explore our code and variables in a real-time, interactive
manner.

Start by examing the \texttt{write} statement from our above program.
It is immediately followed by two stars {\tt *}, which indicate the \emph{output}
method and the \emph{format code}, which specifies how to organize the
data.  The stars ({\tt *}) tell Fortran to use the defaults.  For the {\tt write}
statement, the default output is \emph{standard out} ({\tt stdout}), which is
the terminal window in a Linux environment.  For the {\tt read} statement, the
default input is \emph{standard in} ({\tt stdin}), or the keyboard.
\textbf{The default behavior of Fortran {\tt read} and {\tt write} is to interact
  with the user via the command line interface}.
It is possible to also read/write to and from files (more on this below) or
directly to character variables (which is very useful for building strings!).
Let's look at standard in/out behavior in action:

\begin{verbatim}
program readwrite
  ! Basic hello world program.
  implicit none
  
  integer :: i=0
  
  write(*,*) "Hello!  Please enter an integer:"
  read(*,*) i

  write(*,*) "The number you entered is", i

end program readwrite
\end{verbatim}

Compile and run.  At the {\tt read} statement, our program prompts you to type
in something.  Do so and press enter.  Neat!  The issue, of course, is what
happens if we \emph{don't} type a number?  Our program crashes, and it's not
pretty.  There are ways of handling this, but that's for another day.

Our input and output commands above all use the default formatting approach
(i.e., the {\tt *} as the second argument to {\tt read} and {\tt write}).  This
begs the question: \emph{what is the default behavior?}  This is a complicated
question, because it depends on several things: the compiler you're using, the
number of characters required, the type of values, etc.  Suffice to say, the
default behavior is usually not pretty.  Let's learn how to change that now.

Replace the last {\tt write} line in the above program with this:

\begin{verbatim}
write(*,'(a, 2x, i5.5)') "The number you entered is", i
\end{verbatim}

Run this code twice, once typing in a short number (less than 100,000) and again
with a long number (more than 100,000).
Our {\tt write} behavior is very different!  Under gfortran, the leading space
before writing to screen is now gone.  Our integer has preceding zeros and takes
up exactly 5 characters worth of space.  There are exactly two blank spaces
between our string and our number.  If our number exceeds 5 characters in
length, Fortran spits out asterisks instead of numbers.
We control this behavior with \emph{format descriptors}
(a.k.a. \emph{format codes}), as we do in other languages.
In this example, we're telling Fortran to write out a string of arbitrary
length ({\tt a}), then add two spaces ({\tt 2x}), then print out an
integer using a maximum of five characters and filling empty characters with
zeros ({\tt i5.5}).  You can chain many format codes together to
get tightly formated lines.  You can even insert strings into the format
code proper: \verb_'("I have ", i3, " dogs")'_ would print out
``I have  10 dogs'' (if the integer printed off is the number 10).
Fortran takes these codes as law, not as suggestions- hence the asterisks if
our format code cannot accomodate the values or characters to be written.

Table \ref{tab:codes} summarizes some of the most important format code specifiers
in Fortran.  Many of these are quite similar to those used in other languages, so
there shouldn't be any surprises to seasoned programmers.  The bottom half
provides a symbol key to describe the $r$, $w$, $m$, and $d$ values that
surround each code.  Each of these modifiers are optional.  If you just use
{\tt f}, you'll get a real number using the compiler's default options.
The $r$ specifier allows you to have one code repeated many times.  For
example, if you want to print off ten real numbers, you can just use
{\tt 10f14.4} instead of repeated {\tt f14.4} ten times.

\begin{table}[!h]
  \begin{center}
  \begin{tabular}{|c|c|r|c|}
    \hline
    Code & Use & Example &  Result \\
    \hline
    $r$\textbf{I}$w.m$ & Integers             & \verb_I10.6_ & \verb_|    001234|_\\
    $r$\textbf{F}$w.d$ & Real Numbers         & \verb_F10.4_ & \verb_| 1234.5679|_\\
    $r$\textbf{E}$w.d$ & Engineering Notation & \verb_E10.4_ & \verb_|0.1235E+04|_\\
    $r$\textbf{ES}$w.d$& Scientific Notation  & \verb_ES10.4_& \verb_|1.2346E+03|_\\
    $r$\textbf{G}$w.d$ & Best-Fit Notation    & \verb_G10.4_ & \verb_| 1235.    |_\\
    &                                         & \verb_G10.3_ & \verb_| 0.123E+04|_\\
    $w$\textbf{X} & Add $w$ spaces & \verb_'100',4X,'200'_   & \verb_|100    200|_\\
    $r$\textbf{A}$w$ & Character (string) data& \verb_a10_   & \verb_|abcdefghij|_\\
    \hline
    \multicolumn{4}{|l|}{\textbf{Symbol Key}}\\
    \multicolumn{1}{|c}{$r$} & \multicolumn{3}{l|}{Number of times to repeat code}    \\
    \multicolumn{1}{|c}{$w$} & \multicolumn{3}{l|}{Total width (number of characters)}\\
    \multicolumn{1}{|c}{$d$} & \multicolumn{3}{l|}{Number of digits right of decimal} \\ 
    \multicolumn{1}{|c}{$m$} & \multicolumn{3}{l|}{Minimum number of characters}      \\
    \hline
    
  \end{tabular}
  \end{center}
  \caption{Important format specifiers for the Fortran {\tt write} statement.}
  \label{tab:codes}
\end{table}

groups of codes, strings in format codes, repeating when running out, etc.

There are several ways to emply format codes.
You can place format codes directly in the {\tt write} statements, as we did above.
You can also 

\section{File I/O}

\section{Declaring Variables}

The generic form for declaring variables is as follows:
\begin{verbatim}
type(type specifier) attributes :: variable_name=value
\end{verbatim}
Of the parts above, the only required portions are the \texttt{type} and
\texttt{variable\_name}.  Let's see some examples:

\begin{verbatim}
  integer, parameter :: iCount=3

  real         :: x1, x2=5
  real(kind=8) :: xDoublePrecision=3.14159265358979323846264338
  
  complex :: c1 = (3.141, -3.141)

  logical :: DoPrint = .true.

  character :: letter = 'a'
  character(len=10) :: NameAuthor='Dan' 
  character(len=*), parameter :: NameProgram='test program'
\end{verbatim}

Above, we see the five (and only five!) basic variable types of Fortran90:
\texttt{integers} (numbers with no decimal values), \texttt{reals} (floating-point
values), `\texttt{complex} numbers (with a real and imaginary portion),
\texttt{logicals} (can either be .true. or .false.), and \texttt{characters}.  There
are also \texttt{derived types}, which are Fortran data structures whose
behavior is declared by the user.

Following some of the \texttt{real} statements, we see \emph{kind specifiers}.
These allow us to set the precision of our reals.  A \texttt is,
nominally, a single-precision real, while a \texttt is double precision.
While your mind may snap to the idea of 32- and 64-bit reals, this is not
a well defined standard.  Further, the number of bits dedicated to the
significand and the exponent varies from machine to machine and from
compiler to compiler.  Therefore, while it is good programming practice to
set the \texttt{kind}, it is not machine portable.  A better way is to explicitly
set the precision:
\begin{verbatim}
integer, parameter :: Real8_ = SELECTED_REAL_KIND(p=14, r=200)
real(kind=Real8_)  :: x=10
\end{verbatim}
\texttt{SELECTED\_REAL\_KIND} obtains a kind number corresponding to a floating
point value with precision, \texttt{p}, of 14 (14 decimal digits required) and
range, \texttt{r}, of 200 (200 powers of ten are required in the exponent).

Character len

Following the types and kinds, , we sometimes see attributes.  Attributes set
additional characteristics of the variables.  The \texttt{parameter} attribute
makes a variable unchangeable.  This is very useful when declaring physical
constants.  Other behavior can be set as well, such as if a variable is a input
or output to a subroutine, whether it is an array or allocatable, and others.

Some of our declared values have default values set.  Others do not.  What
are the default values of variables that we did not set explicitly?

implicitly typed variables are BAD!!!

Arrays: indexing defaults to 1-based, but can be anything.

\section{Math \& Logical Operations}

When two numbers of different type or precision are used together in a single
operation, a \emph{mixed-mode arithmatic} situation arises.  What happens when
you multiply an integer and a real number?  What about a single-precision
real and a double precision number?  The answer to these questions are
\emph{machine and compiler dependent}, but there are some ground rules that
generally hold true.  First, when combining an integer and a real, Fortran
will most often convert the integer and avoid loss of precision.  When
combining two reals of different precision, Fortran will typically convert the
value of lesser precision to the kind of higher precision.  Always be
careful when encountering mixed-mode arithmatic.

Here's another tricky situation:
\begin{verbatim}
  real(kind=8) :: x = 5
  write(*,*) x*3.1415
\end{verbatim}
What is the precision of the value written to screen?  What is true above
will hold here as well beholden to the usual caveats concerning different
machines and compiliers.  It is reasonable to expect Fortran to increase the
precision of the unnamed constant value.  Some compilers will complain about
this.  There are a few ways we can remove all doubt, however.
\begin{verbatim}
  integer, parameter :: _DBL=SELECTED_REAL_KIND(p=14,200)
  real(kind=8) :: x = 5
  write(*,*) x*3.1415_8  ! Works only if "8" is a valid kind.
  write(*,*) x*3.1415D0  ! As apposed to 3.1415E0 exponent notation.
  write(*,*) x*3.1415_DBL! ...where _DBL was defined above.
\end{verbatim}

\section{Code Structures}

if-then-else
case statements
do
do while


\section{Subprograms}

\section{Debugging}


\end{document}
